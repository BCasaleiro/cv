\documentclass[11pt,a4paper,sans]{moderncv}

% \usepackage[scale=0.75]{geometry}
\usepackage[utf8]{inputenc}

\usepackage[scale=0.8, top=2cm, bottom=1cm, right=1.5cm, left=1.5cm]{geometry}

\moderncvstyle{banking}
\moderncvcolor{green}
\nopagenumbers{}

\firstname{Bernardo}
\familyname{Casaleiro}

\title{Data Engineer}
\email{bernardocasaleiro@gmail.com}
\homepage{bcasaleiro.github.io}
\social[github][twitter.com/bcasaleiro]{bcasaleiro}

\begin{document}

\vspace*{-1\baselineskip}

\makecvtitle

\vspace*{-2\baselineskip}

%-----------------------------
%	WORK EXPERIENCE SECTION
%-----------------------------
\section{Work Experience}

% -- Meight
\cventry{October 2019 -- Present}{Data Engineer}{\href{https://meight.com}{Meight}}{Remote / Lisbon, Portugal}{}{}
\cvlistitem{Set up and managed a \href{https://cassandra.apache.org/}{Cassandra} cluster of up to 9 nodes, and its backup system.}
\cvlistitem{Implemented the \href{https://pri.meight.com/}{Primeight} Open Source Python package to standardize how people inside Meight interacted with Cassandra.}
\cvlistitem{Set up monitoring and analytics for all services using \href{https://www.elastic.co/what-is/elk-stack}{ELK Stack}, which led to an API success rate of over 99.5\%.}
\cvlistitem{Designed and enforced a plan that reduced AWS costs by 65\% while improving the uptime of all systems.}
\cvlistitem{Implemented and managed multiple ETL pipelines using \href{https://airflow.apache.org}{Apache Airflow}.}
\cvlistitem{Designed and implemented multiple API services using \href{https://fastapi.tiangolo.com/}{FastAPI}.}

\cventry{February 2019 -- January 2020}{Data Scientist}{\href{https://meight.com}{Meight}}{Lisbon, Portugal}{}{}
\cvlistitem{Developed a pipeline to assess the quality of the received data.}
\cvlistitem{Implemented a Complex Event Processor using \href{https://siddhi.io/}{Siddhi} to detect and correct driver mistakes in real-time.}
\cvlistitem{Implemented an ETL pipeline to detect the beginning and end of Trips automatically.}

% -- Muse.ai
\cventry{September 2017 -- December 2018}{Artificial Intelligence Engineer}{\href{https://muse.ai}{muse.ai}}{Lisbon, Portugal}{}{}
\cvlistitem{Designed and developed the company's distributed file system.}
\cvlistitem{Developed Object Recognition, OCR and Action Recognition pipelines and respective curation tools.}
\cvlistitem{Designed and trained a deep learning model to perform OCR using Keras with Tensorflow.}

% -- Stratio
\cventry{May 2015 -- March 2016}{Web Developer}{\href{https://stratioautomotive.com/}{Stratio}}{Coimbra, Portugal}{}{}
\cvlistitem{Full Stack Web Development using Laravel, Node.js, AngularJS and jQuery.}
\cvlistitem{Worked on an Android application that helps users pick a public transportation route for the city of Coimbra.}

% -- Jeknowledge
\cventry{October 2013 -- May 2015}{Junior Developer}{\href{https://jeknowledge.pt}{JeKnowledge}}{Coimbra, Portugal}{}{}

%-----------------------------
%	EDUCATION SECTION
%-----------------------------
\section{Education}

\cventry{2016 -- 2018}{Masters in Information Systems and Computer Engineering}{\href{https://tecnico.ulisboa.pt/en/}{Instituto Superior Técnico}}{Lisbon, Portugal}{}{Algorithms and Intelligent Systems Focus}

\cventry{2013 -- 2016}{Bachelor in Software Engineering}{\href{http://www.uc.pt/en}{University of Coimbra}}{Coimbra, Portugal}{}{}

% -- Masters Thesis
\section{Masters Thesis}

\cvitem{Title}{\emph{Morphosyntactic Label Disambiguation}}
\cvitem{Supervisors}{\href{http://web.tecnico.ulisboa.pt/bruno.g.martins}{Professor Bruno Martins} \& \href{https://www.l2f.inesc-id.pt/wiki/index.php/Nuno_Mamede}{Professor Nuno Mamede}}
\cvitem{Description}{This thesis explored the use of hand-crafted rules in combination with Recurrent Neural Networks and Conditional Random Fields to disambiguate 11 different morphosyntactic labels (e.g. noun, verb, ...) for each word.}

%-----------------------------
%	LANGUAGE SECTION
%-----------------------------
\section{Languages}
\cvitemwithcomment{Portuguese}{Native language}{}
\cvitemwithcomment{English}{Full professional proficiency}{}

\end{document}
